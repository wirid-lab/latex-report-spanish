	%%%%%%%%%%%%%%%%%%%%%%%%%%%%%%%%%%%%%%%%%%%%%%%%%%%%%%%%%%%%%%%%%%%%%%%%%%%%%%%%
	 % CONTENIDO
	 
	 
\documentclass[12pt,spanish]{article}
\usepackage[spanish]{babel}
\selectlanguage{spanish}
\usepackage[T1]{fontenc}  
\usepackage{textcomp}  
\usepackage{lmodern} 


%\usepackage{biblatex}
%\usepackage[style=numeric-comp]{biblatex}
\usepackage{amsmath}
\usepackage{geometry}
\usepackage{pdflscape}
\usepackage{tabularx}
\usepackage{hyperref}
\usepackage{subfig}

\usepackage[font=small,labelfont=bf]{caption}

\usepackage{tikz,lipsum,lmodern}
\usepackage[most]{tcolorbox}
\usepackage{caption}
\captionsetup{font=small,labelfont=bf}
\newcommand{\source}[1]{\vspace{-10pt}\caption*{ Fuente: {#1}} }

%INCLUIR PAQUETES
\usepackage{url}
\usepackage{amsmath}
\usepackage{graphicx}
\graphicspath{{images/}} %carpeta de imagenes
\usepackage{parskip}
\usepackage{fancyhdr}
\usepackage{vmargin}
\usepackage{pdfpages}
\usepackage{lipsum,appendix}
\usepackage{enumitem} % Customize lists
\setlist{nolistsep} % Reduce spacing between bullet points and
\usepackage{color}
\usepackage{booktabs}
\usepackage{xcolor,colortbl}
\usepackage{float} 
\usepackage{booktabs} % Required for nicer horizontal rules in tables



\usepackage{subfigure} % subfiguras
\usepackage{lipsum}  


%RE CONFIGURANDO LOS COMANDOS

%Cambiando el nombre de los apendices a Anexos
\renewcommand{\appendixtocname}{Anexos}
\renewcommand{\appendixpagename}{Anexos}

%Cambiar Nombres a diferentes contenidos
\renewcommand{\listfigurename}{Lista De Figuras}
\renewcommand{\listtablename}{Lista de Tablas}
\renewcommand{\contentsname}{Lista de Contenidos}
\renewcommand{\figurename}{Figura}
\renewcommand{\tablename}{Tabla} 


\def\captionsspanish{%
	\def\prefacename{Prefacio}%
	\def\refname{Referencias}%
	\def\abstractname{Resumen}%
	\def\bibname{Bibliografía}%
	\def\chaptername{Capítulo}%
	\def\contentsname{Contenido}

}







%AJUSTANDO LAS MARGENES DEL DOCUMENTO
\setmarginsrb{2.5 cm}{1 cm}{2.5 cm}{2 cm}{0.5 cm}{0.8 cm}{1 cm}{1.0 cm}


%----------------------------------------------------------------------------------------
%	BIBLIOGRAFIA E INDICE 
%----------------------------------------------------------------------------------------

%\usepackage[style=ieee,citestyle=numeric,sorting=none,sortcites=true,autopunct=true,babel=hyphen,hyperref=true,abbreviate=false,backref=true,backend=biber]{biblatex}
%%\addbibresource{bibliography.bib} % BibTeX bibliography file
%\addbibresource{referencias.bib}
%\defbibheading{bibempty}{}
%\usepackage{calc} % For simpler calculation - used for spacing the index letter headings correctly
%\usepackage{makeidx} % Required to make an index
%\makeindex % Tells LaTeX to create the files required for indexing



%----------------------------------------------------------------------------------------
% FUENTES
%----------------------------------------------------------------------------------------
%\usepackage{fontspec}
%\setmainfont{fonts/texgyreadventor/texgyreadventor-regular.ttf}


%----------------------------------------------------------------------------------------
%	ESTILO AL HEADER Y FOOTER
%----------------------------------------------------------------------------------------
\usepackage{xcolor}
\usepackage{sectsty}
\pagestyle{fancy}

\newcommand{\colorplantilla}{054D77}

%%%  COLOR AZUL CLARO 		-----  	197DB9
%%%  COLOR AZUL OSCURO 		-----	054D77
%%%  COLOR VERDE			-----  	4D8D39
%%%  COLOR VERDE OSCURO		-----  	1A4F09
%%%  COLOR NARANJA			-----  	CB6017

%%%%%%%%%%%%%%%%%%%% SE ASIGNA COLOR DEL ENCABEZADO %%%%%%%%%%%%%%%%%
\renewcommand{\headrulewidth}{1pt}
\renewcommand{\headrule}{\hbox to\headwidth{%
		\color[HTML]{\colorplantilla}\leaders\hrule height \headrulewidth\hfill}}


\fancyhf{}
\setlength\headheight{26pt}

%Imágen de cabecera

%\rhead{\includegraphics[height=1.5cm]{logo.png}}
\lhead{\scriptsize{Nombre_proyecto... \thetitle}.Descripcion_1}

%\lfoot{\centering
%	Grupo de Investigación en Seguridad y Sistemas de Comunicación (GISSIC) \\ \thepage
%}

 \fancyfoot[RO,LE]{\thepage} 



%%%%%%%%%%%%%%%%% SE ASIGNA COLOR A LOS TEXTOS %%%%%%%%%%%%%%%%%%%%%%%%%%%%
\sectionfont{\color[HTML]{\colorplantilla}}  % Cambia Color a los Titulos de Seccion
\subsectionfont{\color[HTML]{\colorplantilla}}  %Cambia color Subseccion
\subsubsectionfont{\color[HTML]{\colorplantilla}}  % Cambia color SUB SUB SECCION

%\arrayrulecolor[HTML]{\colorplantilla}
%%%%%%%VINCULAR TODO LO QUE CONTENGA EL DOCUMENTO EN LA TABLA DE CONTENIDOS
\usepackage{hyperref}
\hypersetup{
	colorlinks=true, %set true if you want colored links
	linktoc=all,     %set to all if you want both sections and subsections linked
	linkcolor=[HTML]{\colorplantilla},  %choose some color if you want links to stand out
	urlcolor  = [HTML]{\colorplantilla},
	citecolor = black,
	anchorcolor = blue
}



 %%CONTIENE TODOS LOS ESTILOS DEL DOCUMENTO

\usepackage{listings}
\usepackage{xcolor}
%New colors defined below
\definecolor{codegreen}{rgb}{0,0.6,0}
\definecolor{codegray}{rgb}{0.5,0.5,0.5}
\definecolor{codepurple}{rgb}{0.58,0,0.82}
\definecolor{backcolour}{rgb}{255,255,255}
\definecolor{light-gray}{gray}{0.95} %the shade of grey that stack 


\lstset{
    backgroundcolor = \color{light-gray},
	basicstyle=\ttfamily\fontfamily{iwona}\tiny\scriptsize,
	frame=single,
	showspaces=false, % show spaces adding particular underscores
	showstringspaces=false, % underline spaces within strings
	showtabs=false, % show tabs within strings adding particular underscores
	tabsize=1, % sets default tabsize to 2 spaces
	captionpos=true, % sets the caption-position to bottom
	breaklines=true, % sets automatic line breaking
	breakatwhitespace=true,
	breakindent=1pt,
	columns=fullflexible,
	literate = {-}{-}1 {*}{*}1 {"}{"}1   , % <-
    keywordstyle=\color{blue},
}



\fancyhf{}
\setlength\headheight{26pt}
\rhead{\includegraphics[height=1cm]{logoWirid-LAB.png}}
\lhead{\scriptsize{Informe de entrega}}

\lfoot{\centering
	Grupo de Investigación en Seguridad y Sistemas de Comunicaciones . GISSIC - UMNG \\ \thepage
}


%%%%%%%%%%%%% NOMBRE DEL DOCUMENTO Y AUTOR %%%%%%%%%%%%%%

\title{Titulo del documento a genera  }

\author{ Grupo de Investigación en Seguridad y Sistemas de Comunicación } 


%\bibliography{references}


%\date{\today}

\makeatletter
\let\thetitle\@title
\let\theauthor\@author
\let\thedate\@date
\makeatother

\usepackage[utf8]{inputenc}
\usepackage{amsmath}
\usepackage{lscape}
\usepackage{pdfpages}
%https://ondahostil.wordpress.com/2017/06/14/lo-que-he-aprendido-marcas-de-agua-en-latex/

%\usepackage{draftwatermark}
%\SetWatermarkText{\textsc{Confidencial}} % por defecto Draft 
%\SetWatermarkScale{0.8} % para que cubra toda la página
%\SetWatermarkColor[rgb]{1,0,0} % por defecto gris claro
%\SetWatermarkAngle{55} % respecto a la horizontal



\begin{document}
	
	%----------------------------------------------------------------------------------------
	%%	PORTADA
	%----------------------------------------------------------------------------------------
	
	\begin{titlepage}
		\centering
		\vspace*{0.5 cm}
		\includegraphics[height=5cm]{LogoUMNG.png}\\[1.0 cm]	% University Logo
		
		\textsc{\LARGE Universidad Militar Nueva Granada  }\\[0.5 cm]	% University Name
		
		\textsc{\large Grupo de Investigación XXX  }\\[1 cm]	% University Name
		
		
		\textsc{\large Nombre Materia / Proyecto  }\\[1 cm]				% Course Name
		
		%\textsc{\large Código UMNG del proyecto: \textbf{ING 2113}}\\[0.5 cm]				% Course Code
		
	%	\textsc{\large Proyecto de Investigación  INV ING XX \\ Año de ejecución: 2018 }\\[0.5 cm] 
		
		\textsc{\large Laboratorio WIRID -LAB  }\\[0.5 cm] 
		
		\rule{\linewidth}{0.2 mm} \\[0.4 cm]
		{ \huge \bfseries \thetitle}\\
		\rule{\linewidth}{0.2 mm} \\[1.5 cm]
		
	
	    \textsc{\large Autor 1 \\
	    Autor 2 }\\[1cm]
		
		\large{\textit{email@email.com \\ email2@email.com}}\\[1cm]
			
		\textsc{\large Fecha }\\[1.5 cm]
		
		\begin{minipage}{0.7\textwidth}
			\begin{flushleft} \large
			%	\emph{Autores:}\\
			%	\theauthor
				
			\end{flushleft}
		\end{minipage}~
		
		\begin{minipage}{0.4\textwidth}
			\begin{flushright} \large
				 		
			\end{flushright}
		
		\end{minipage}\\[2 cm]
		
	%	{\large \thedate}\\[2 cm]
		
		\vfill
				
	\end{titlepage}
		
	\pagenumbering{gobble}
	
	%----------------------------------------------------------------------------------------
	%%	DECLARACIÓN
	%----------------------------------------------------------------------------------------
	
	\thispagestyle{empty}
	\vspace*{\fill}
	\begingroup 




% Si es necesario que este documento lo firme alguien aqui va. \\
	

% 	\hspace{0.1em}
% 	\vspace{15em}
% 	\noindent\begin{tabular}{ll}
% 		\makebox[4.5in]{\hrulefill}\\ 
% 		%& \makebox[3.5in]{\hrulefill}\\
% 	     Nombre de la persona quien firma 
% 	    % & Fecha\\[8ex]
% 		\\
% 		\\
		
	
	
	\end{tabular}
	
	\endgroup
	
%	\centering Marzo 16 de 2018
	
	\vspace*{\fill}
	
	%----------------------------------------------------------------------------------------
	%%	INCLUIR TABLA DE CONTENIDOS E INICIAR LAS PAGINACION DE LAS HOJAS 
	%----------------------------------------------------------------------------------------
	
	\newpage
	
	%%%%%%%%%%%%%%%%%%%%%%%%%%%%%%%%%%%%%%%%%%%%%%%%%%%%%%%%%%%%%%%%%%%%%%%%%%%%%%%%%%%%%%%%%
	
	\tableofcontents	
	\pagebreak
	\pagenumbering{arabic}
	

%----------------------------------------------------------------------------------------
%%	INCIO DEL DOCUMENTO
%----------------------------------------------------------------------------------------	

\section{Introducción}


Lorem ipsum dolor sit amet, consectetur adipiscing elit. Proin eget nulla eget dui tincidunt scelerisque. Quisque gravida mi nec nibh euismod dictum. Curabitur ac lorem nisi. Quisque vitae laoreet felis. Pellentesque in lacus sollicitudin, imperdiet libero in, venenatis dolor. Phasellus feugiat velit aliquet porta elementum. Vivamus eget arcu metus. Vestibulum id erat id orci blandit posuere aliquet ut ipsum.

In commodo nulla quis pretium pretium. Etiam ornare neque nunc, ac faucibus felis posuere sit amet. Maecenas sodales est id rhoncus ullamcorper. Donec nec sapien in nibh eleifend placerat non elementum orci. Donec convallis auctor facilisis. Fusce finibus quam vel purus tempus suscipit. Nam non lorem ut erat convallis hendrerit id in ipsum. Cras aliquam ultricies risus non rutrum. Duis at imperdiet magna. Aenean maximus orci tortor, at laoreet purus lobortis vel. Vivamus lacinia vestibulum elit at maximus.

Nunc ac posuere erat. Interdum et malesuada fames ac ante ipsum primis in faucibus. In tempus, elit eu mattis imperdiet, nulla ligula consectetur tortor, at lobortis sem odio id lacus. Pellentesque rhoncus urna id enim venenatis semper. Vestibulum lacinia mauris sit amet elit consequat, nec congue erat laoreet. Aenean leo ligula, ultricies vel magna non, blandit sodales dolor. Sed sit amet ultricies lorem. Sed hendrerit, lorem eget semper malesuada, lacus justo sagittis magna, ut viverra dolor libero non diam. Donec lorem ante, ullamcorper eget tortor mollis, aliquet dignissim lacus. Sed accumsan, dui quis consequat rhoncus, turpis quam ultricies lorem, laoreet condimentum elit purus ac turpis. In et massa laoreet, imperdiet lorem nec, malesuada odio. In hac habitasse platea dictumst. Cras rutrum nisl dui, cursus condimentum ipsum molestie ut. Proin sit amet leo et arcu ultrices congue at vel est.

Mauris scelerisque dignissim nibh, id fringilla risus laoreet sit amet. Orci varius natoque penatibus et magnis dis parturient montes, nascetur ridiculus mus. Donec massa ipsum, commodo in nisi sed, pellentesque vestibulum dui. Nullam vitae nunc dictum, luctus purus vitae, hendrerit mi. Mauris sed justo in nisi tincidunt semper vitae placerat magna. Vestibulum rutrum dignissim condimentum. Nunc posuere bibendum lacinia. Aenean et dui non odio cursus tempor. Quisque rutrum nunc a lorem malesuada pharetra. Nullam condimentum at nisl vitae cursus. Quisque tempus lorem et urna venenatis semper. Maecenas aliquet porttitor lectus in elementum. Proin non ligula mauris. In non sollicitudin mauris.




\section{Titulo 1}



%%%%%%%%%%%  SUBSECTION %%%%%%%%%%%%%%%%%% 
\subsection{Subtitulo 1 }

Lorem ipsum dolor sit amet, consectetur adipiscing elit. Proin eget nulla eget dui tincidunt scelerisque. Quisque gravida mi nec nibh euismod dictum. Curabitur ac lorem nisi. Quisque vitae laoreet felis. Pellentesque in lacus sollicitudin, imperdiet libero in, venenatis dolor. Phasellus feugiat velit aliquet porta elementum. Vivamus eget arcu metus. Vestibulum id erat id orci blandit posuere aliquet ut ipsum. 
\begin{lstlisting}[ basicstyle=\normal] 
lb
\end{lstlisting}




%%%%%%%%%%%  SUBSECTION %%%%%%%%%%%%%%%%%% 
\subsection{Subtitulo 2}
Una vez creada la aplicación es necesario agregar las diferentes tablas que componen la base de datos, estas tablas en Loopback se conocen como modelos. Para crear los modelos es necesario ingersar a la carpeta de la aplicación y ejecutar la siguiente linea de comando:

\begin{lstlisting}[ basicstyle=\normal] 
lb model
\end{lstlisting}


\begin{itemize}
\item  1
\item 2
\item 3
\item 4
\item 5
\item 6
\end{itemize}

Lorem ipsum dolor sit amet, consectetur adipiscing elit. Proin eget nulla eget dui tincidunt scelerisque. Quisque gravida mi nec nibh euismod dictum. Curabitur ac lorem nisi. Quisque vitae laoreet felis. Pellentesque in lacus sollicitudin, imperdiet libero in, venenatis dolor. Phasellus feugiat velit aliquet porta elementum. Vivamus eget arcu metus. Vestibulum id erat id orci blandit posuere aliquet ut ipsum.   \cite{PaddlePaddledevelopers2017}

\newpage
\section{Titulo 2 }
\subsection{Anexo 1}
Lorem ipsum dolor sit amet, consectetur adipiscing elit. Proin eget nulla eget dui tincidunt scelerisque. Quisque gravida mi nec nibh euismod dictum. Curabitur ac lorem nisi. Quisque vitae laoreet felis. Pellentesque in lacus sollicitudin, imperdiet libero in, venenatis dolor. Phasellus feugiat velit aliquet porta elementum. Vivamus eget arcu metus. Vestibulum id erat id orci blandit posuere aliquet ut ipsum. 


\begin{itemize}
\item \textbf{text:} Lorem ipsum dolor sit amet, consectetur adipiscing elit. Proin eget nulla eget dui tincidunt scelerisque. Quisque gravida mi nec nibh euismod dictum. Curabitur ac lorem nisi. Quisque vitae laoreet felis. Pellentesque in lacus sollicitudin, imperdiet libero in, venenatis dolor. Phasellus feugiat velit aliquet porta elementum. Vivamus eget arcu metus. Vestibulum id erat id orci blandit posuere aliquet ut ipsum. . \\ 
\item \textbf{text:} Lorem ipsum dolor sit amet, consectetur adipiscing elit. Proin eget nulla eget dui tincidunt scelerisque. Quisque gravida mi nec nibh euismod dictum. Curabitur ac lorem nisi. Quisque vitae laoreet felis. Pellentesque in lacus sollicitudin, imperdiet libero in, venenatis dolor. Phasellus feugiat velit aliquet porta elementum. Vivamus eget arcu metus. Vestibulum id erat id orci blandit posuere aliquet ut ipsum. 
\end{itemize}

\newpage




\newpage
\section{Bibliografía}
\bibliographystyle{IEEEtran}
\bibliography{Ref}

%\newpage
%\section{Anexos}
%\input{5_Anexos.tex}

	%\bibliographystyle{IEEEtran}




	
%	\nocite{*}
%	\printbibliography[heading=subbibliography,heading=none]
%	\newpage
%	\addappheadtotoc 
%	\appendix    
%	
%		\includepdf[pages={1},scale=.94,offset={2.5cm -4cm},pagecommand={\section{Documento PDF}\label{anexo:codigo_pdf} }]{archivo.pdf}
%		\includepdf[pages={2-},scale=.94,offset={2.5cm -4cm},pagecommand={\thispagestyle{fancy}}]{archivo.pdf}
%		
%		\section{Otro Anexo}\label{anexo:codigo_matlab}
%		
%		\input{anexo_1.tex}
	
	
	
\end{document}